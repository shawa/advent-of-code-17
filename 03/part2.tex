% Created 2017-12-06 Wed 20:50
% Intended LaTeX compiler: pdflatex
\documentclass[11pt]{article}
\usepackage[utf8]{inputenc}
\usepackage[T1]{fontenc}
\usepackage{graphicx}
\usepackage{grffile}
\usepackage{longtable}
\usepackage{wrapfig}
\usepackage{rotating}
\usepackage[normalem]{ulem}
\usepackage{amsmath}
\usepackage{textcomp}
\usepackage{amssymb}
\usepackage{capt-of}
\usepackage{hyperref}
\usepackage{mathtools}
\DeclarePairedDelimiter{\ceil}{\lceil}{\rceil}
\author{Andrew Shaw <shawa1@tcd.ie>}
\date{\today}
\title{Computing \(\ell_1\) distance to origin on a Spiral over the Natural Numbers}
\hypersetup{
 pdfauthor={Andrew Shaw <shawa1@tcd.ie>},
 pdftitle={Computing \(\ell_1\) distance to origin on a Spiral over the Natural Numbers},
 pdfkeywords={},
 pdfsubject={},
 pdfcreator={Emacs 25.2.1 (Org mode 9.0.8)}, 
 pdflang={English}}
\begin{document}

\maketitle
\tableofcontents

\section{Abstract}
\label{sec:orga7814cd}
We present a coordinate space for addressing elements on the Integer Spiral
which is rotationally symmertric 90\textdegree{} about the origin (1). We develop and
present useful supporting formulae, in terms of some integer \(n\), for computing
the coordinates of \(n\) in this space, with a view to computing the \(\ell_1\)
(taxicab) distance from \(n\) to the origin.

\section{Introduction}
\label{sec:orge08f20c}
We define a Spiral over some ordered set \(S\) to be an arrangement of the
elements of \(S\) in an \(N \times N\) grid - where \(N\) is odd - such that the least
element of \(S\) is in the centre, with the elements arranged in a
counter-clockwise fashion, spiralling outward from the centre.

An example of one such Spiral is the Spiral over the Natural numbers, which we
will denote as \(\mathcal{N}\):

\begin{verbatim}
DIAGRAM OF THE SPIRAL
\end{verbatim}

As with any ordered set, it will prove useful to develop a distance metric for
different elements of \(\mathcal{N}\), not least for solving part one of day three
of the 2017 edition of the \emph{Advent of Code}. Per the problem, we will concern
ourselves with the \(\ell_1\) of some element \(n \in \mathcal{N}\) to the origin, 1.


\subsection{Shells in the Spiral}
\label{sec:org42624b0}
When dealing with elements in \(\mathcal{N}\), it will prove useful to partition
\(\mathcal{N}\) into subsets radiating outward from the origin, which we will call
\emph{shells}, denoted as \(S_n\), where \(n\) is the number of shells between \(S_n\) and
the origin. See, for example, \(S_0\), \(S_1\), and \(S_5\), below, highlighted each
in red, white and green, respectively:

\begin{verbatim}
Highlighted DIAGRAM OF THE SPIRAL
\end{verbatim}

\subsection{Boundary numbers}
\label{sec:org7992758}
We define the \emph{boundary number} of a shell to be the greatest number in that
shell, so-called as it marks the boundary between two successive shells when
following the spiral from the center.

We can see by inspection that boundary numbers of \(S_0\), \(S_1\), and \(S_3\) are 0,
9, 25, and 49. Noting that this is simply the sequence of squares of odd
numbers, in general, the boundary number for \(S_n\) is
simply

$$ b(n) = (2n + 1)^2 $$

The inverse \footnote{\(b\) is invertible as of course we are ignoring the
case where \(n\) is negative, as the notion of ``shell -1'' is clearly absurd} of this function 

$$ b^{-1}(S) = \frac{1}{2} (\sqrt{x} - 1) $$

will, given a boundary number, yield its shell number.


Further treatment of this function is required should we wish to find the shell
number of \emph{any} number, not just a boundary number.

A commonly known method of finding the \emph{next closest} square of some real number
\(x\) is to compute \(\ceil*{\sqrt{x}} ^ 2\).
\end{document}
